\documentclass{article}
\usepackage[utf8]{inputenc}
\usepackage{hyperref}
\usepackage{minted}

\title{CRS Implementation (zei-library)}
%\author{serge.galois }
\date{August 2021}

\begin{document}

\maketitle

%\section{CRS Implementation (zei-library)}

\vspace{1cm}

In our implementation the CRS is based in KZG10 polynomial commitments over  BLS12-381 elliptic curv, all this to get a PlonK inplementation.\\


“Due to real-world deployments of zk-SNARKs, it has become of significant interest to have the structured reference string (SRS) be constructible in a “universal and updat- able” fashion. Meaning that the same SRS can be used for statements about all circuits of a certain bounded size; and that at any point in time the SRS can be updated by a new party, such that the honesty of only one party from all updaters up to that point is required for soundness. For brevity, let us call a zk-SNARK with such a setup process universal…”\footnote{Taken from the PlonK paper's introduction.}\\


If Fp = F is a field of prime order, we can denote by $F_{<d}[X]$ the set of univariate polynomials over F of degree smaller than d. The implementation of KZG polynomial commitment scheme based on \url{https://www.iacr.org/archive/asiacrypt2010/6477178/6477178.pdf} it relies on a bilinear map $e:G_1 \times G_2 \longrightarrow G_t$, where $G_1,G_2,G_t$ are cyclic groups of prime order $p$, and $g_1$ is a generator fro $G_1$ and $g_2$ is a generator for $G_2$.\\

The operations of the scheme are as follows:\\

\begin{itemize}
    \item SRS generation(n: max polynomial degree)\\
    \begin{itemize}
        \item  Pick a random scalar s in $F_p$\\
        \item Compute public\_parameter\_group\_1:= $(g_1,g_1^s,g_1^{s^2},...,g_1^{s^n})$\\
        \item Compute public\_parameter\_group\_2:= $(g_2,g_2^s)$\\
        \item return (public\_parameter\_group\_1,public\_parameter\_group\_2)\\
    \end{itemize}
    
\end{itemize}
    This is the function in our implementations that takes care of this.\\
    \url{https://github.com/FindoraNetwork/zei/blob/9951a0364f5f4228d3fa8ac1cabf679b0a881a55/poly-iops/src/commitments/kzg_poly_com.rs#L149}\\
    
\begin{itemize}
    \item Commit($P$: polynomial)\\
    \begin{itemize}
        \item let $P(x) = a0 + a1X + a2X^2 + ...+ a_nX^n$\\
        \item let $C := g1^{P(s)} = \pi_{i=0}^n (g_i^{s^i})^{a_i}$\\
        \item return $C$\\ 
    \end{itemize}


    \item Prove eval($P$:polynomial, $x$: evaluation point)
    \begin{itemize}
        \item Let $y=P(x)$\\
        \item Compute Q(X) = (P(X)-P(x))/(X-x)   if indeed $y==P(x)$ then $(X-x)|P(X)-y$\\
        \item return $g1^{Q(s)}$\\
    \end{itemize}

    \item Verify eval($C$: commitment, $x$: evaluation point, $y$: evaluation of $P$ on $x$, proof: proof of evaluation). The goal of this verification procedure is to check that indeed $P(X)-y=Q(X)(X-x)$ using pairings. Check that $e(C/g1^y,g2) == e(proof,g2^s/g2^x)$\\

\end{itemize}



With help of this we could prove that $f(7) = 57$, for a polynomial without revel who exactly is the polynomial\\


The max degree of the polynomial that can be committed is computed according to the number of constraints in the system by this function “build\_multi\_xfr\_cs”\\

\begin{minted}{rust}
/// A prover can provide honest `secret_inputs` and obtain
///the cs witness by calling `cs.get_and_clear_witness()`.

/// One provides an empty secret_inputs to get the constraint 
///system `cs` for verification only.

/// Returns the constraint system (and associated number of 
///constraints) for a multi-inputs/outputs transaction.

\end{minted}

\url{https://github.com/FindoraNetwork/zei/blob/9951a0364f5f4228d3fa8ac1cabf679b0a881a55/zei_api/src/anon_xfr/circuits.rs#L191}\\

The process pad the outcome number to the minimum power of two  greater or equal to the actual number of constraints (for example, if the outcome number of constraints is 250 it is set to $256 = 2^8$)\\

\begin{minted}{rust}
/// Pad the number of constraints to a power of two.
\end{minted}

\url{https://github.com/FindoraNetwork/zei/blob/9951a0364f5f4228d3fa8ac1cabf679b0a881a55/poly-iops/src/plonk/turbo_plonk_cs/mod.rs#L537}\\


The CRS is part of the PublicParams structure who has the following inputs and outputs:\\

The “impl PublicParams” needs this parameters\\

\begin{minted}{rust}
impl UserParams {
    pub fn new(
        n_payers: usize,
        n_payees: usize,
        tree_depth: Option<usize>,
        bp_num_gens: usize,
    ) -> UserParams {

And returns a structure like this  

pub struct UserParams {
    pub bp_params: PublicParams,
    pub pcs: KZGCommitmentSchemeBLS,
    pub cs: TurboPlonkCS,
    pub prover_params: ProverParams<KZGCommitmentSchemeBLS>,
}
\end{minted}


\url{https://github.com/FindoraNetwork/zei/blob/9951a0364f5f4228d3fa8ac1cabf679b0a881a55/zei_api/src/setup.rs#L121}\\


Where the field pub pcs is the substructure that contains the CRS to perform the KZG10 polynomial commitments based on the BLS12-381 elliptic curve.\\

If the public params were generated before they are stored as binaries files in path “zei/zei\_api/data”, so the first step is verify if pior public parameters exists in affirmative case they are load from the file, otherwise they are generated and stored in the path for future instances.\\  




\end{document}
